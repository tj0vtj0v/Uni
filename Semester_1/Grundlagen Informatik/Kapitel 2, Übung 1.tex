\documentclass[a4paper,onecolumn,pdftex]{report}

\usepackage{enumitem}
\usepackage[p,osf]{scholax}
\usepackage{amsmath}
\usepackage[margin=1in]{geometry}

\fontfamily{scholax}
\overfullrule=0pt

\begin{document}
    \author{Noah Tjorven Burdorf}
    \title{Kapitel 2, \"Ubung 1 \\ Grundlagen Informatik \\ Vorlesung vom 11.10.2023}
    \date{23/10/11}
    \maketitle

    \begin{enumerate}
        \item Stellen Sie die Wahrheitstabelle folgender aussagenlogischer Formeln auf. 
        \begin{enumerate}
            \item $(\neg A \wedge \neg B) \lor  C$ \\
            \begin{tabular}{|c|c|c|c|c|} \hline
                A & B & C & $\neg A \wedge \neg B$ & $(\neg A \wedge \neg B) \lor  C$ \\ \hline
                F&F&F&T&T \\ \hline
                F&F&T&T&T \\ \hline
                F&T&T&F&T \\ \hline
                F&T&F&F&F \\ \hline
                T&F&F&F&F \\ \hline
                T&F&T&F&T \\ \hline
                T&T&F&F&F \\ \hline
                T&T&T&F&T \\ \hline
            \end{tabular}

            \item $\neg A \lor \neg B \lor C$ \\
            \begin{tabular}{|c|c|c|c|c|c|} \hline
                A&B&C&$\neg A$&$\neg B$&$\neg A \lor \neg B \lor C$ \\ \hline
                F&F&F&T&T&T \\ \hline
                F&F&T&T&T&T \\ \hline
                F&T&F&T&F&T \\ \hline
                F&T&T&T&F&T \\ \hline
                T&F&F&F&T&T \\ \hline
                T&F&T&F&T&T \\ \hline
                T&T&F&F&F&F \\ \hline
                T&T&T&F&F&T \\ \hline
            \end{tabular}
        \end{enumerate}

        \vspace{50px}
        \item Stellen Sie die Wahrheitstabelle folgender aussagenlogischer Formel auf. Reduzieren Sie vorher die Anzahl der benötigten Zwischenaussagen durch Anwendung von der Regel nach DeMorgan.  \\
        $\neg (\neg A \wedge B) \lor \neg C \equiv A \lor \neg B \lor \neg C$ \\
        \begin{tabular}{|c|c|c|c|c|c|} \hline
            A&B&C&$\neg B$&$\neg C$&$A \lor \neg B \lor \neg C$ \\ \hline
            F&F&F&T&T&T \\ \hline
            F&F&T&T&F&T \\ \hline
            F&T&F&F&T&T \\ \hline
            F&T&T&F&F&F \\ \hline
            T&F&F&T&T&T \\ \hline
            T&F&T&T&F&T \\ \hline
            T&T&F&F&T&T \\ \hline
            T&T&T&F&F&T \\ \hline
        \end{tabular}

        \newpage
        \item Vereinfachen Sie die folgenden Ausdrücke, bis nur noch die booleschen Operatoren $\lor$ und $\neg$ vorkommen:
        \begin{enumerate}
            \item $(\neg A \wedge \neg B) \lor C$
            \begin{align*}
                (\neg A \wedge \neg B) \lor C &\equiv \neg (A \lor B) \lor C
            \end{align*}

            \item $\neg (A \wedge \neg (B \Rightarrow C))$
            \begin{align*}
                \neg (A \wedge \neg (B \Rightarrow C)) &\equiv \neg A \lor (B \Rightarrow C) \\
                &\equiv \neg A \lor (\neg B \lor C) \\
                &\equiv \neg A \lor \neg B \lor C
            \end{align*}

            \item $\neg ((A \wedge \neg B) \wedge \neg (C \Rightarrow D))$
            \begin{align*}
                \neg ((A \wedge \neg B) \wedge \neg (C \Rightarrow D)) &\equiv \neg (A \wedge \neg B) \lor (C \Rightarrow D) \\
                &\equiv (\neg A \lor B) \lor (\neg C \lor D) \\
                &\equiv \neg A \lor B \lor \neg C \lor D
            \end{align*}
        \end{enumerate}
    
        \vspace{50px}
        \item Stellen Sie fest, ob die folgenden aussagenlogischen Formeln Tautologien sind: Nutzen Sie dazu keine Wertetabelle/Wahrheitstabelle sondern erklären Sie Ihre Feststellung! Falls es keine Tautologie ist, genügt eine Belegung mit dem Ergebnis „falsch“. 
        \begin{enumerate}
            \item $(A \wedge \neg A) \Rightarrow (B \lor C)$ \\
            Ist eine Tautologie da $(A \wedge \neg A)$ eine Kontradiktion ist, und somit immer falsch. \\
            Woraufhin der erste Teil der Implikation immer falsch ist, weswegen die Aussage immer wahr ist.
            
            \item $(A \wedge B) \Leftrightarrow (A \lor B)$ \\
            falsch
        \end{enumerate}
    
        \vspace{50px}
        \item Stellen Sie fest, ob die folgenden Aussagenlogischen Formeln Kontradiktionen sind: 
        \begin{enumerate}
            \item $A \wedge \neg A$ \\
            Ist eine Kontradiktion, da immer $A$ oder $\neg A$ faslch ist.
            
            \item $(A \wedge \neg B) \Leftrightarrow (\neg A \lor B)$ \\
            Ist eine Kontradiktion, da immer eine Seite falsch und eine Seite wahr ist, und eine Äquivalenz vorliegt. \\
            Da eine Äquivalenz immer falsch ist wenn beide Seiten unterschiedliche Wahrheitswerte haben, ist die gesamte Aussage falsch.
        \end{enumerate}
    
        \newpage
        \item Zeigen Sie folgende Äquivalenz durch Umformung der linken Seite. Notieren Sie bei jedem Schritt alle Rechenregeln die Sie anwenden.
        \begin{align}
            (C \wedge T) \lor \neg (\neg A \Rightarrow B) \lor (\neg A \wedge \neg C) \lor \neg (\neg C) \lor (\neg A \wedge A) &\equiv (A \lor (B \wedge C)) \Rightarrow C \\
            C \lor \neg (A \lor B) \lor \neg (A \lor C) \lor C \lor K &\equiv \\
            C \lor \neg (A \lor B) \lor \neg (A \lor C) \lor C &\equiv \\
            C \lor (\neg A \wedge \neg B) \lor (\neg A \wedge \neg C) \lor C &\equiv  \\
            ((\neg A \wedge \neg B) \lor (\neg A \wedge \neg C)) \lor C &\equiv \\
            (\neg A \wedge (\neg B \lor  \neg C)) \lor C &\equiv \\
            \neg A \wedge \neg (B \wedge C) \lor C &\equiv \\
            \neg (A \lor (B \wedge C)) \lor C &\equiv (A \lor (B \wedge C)) \Rightarrow C
        \end{align}

        \vspace{20px}
        \setlist[enumerate]{label={\arabic*.}}
        \begin{enumerate}
            \item Originale Aufgabe
            \item K\"urzungsregel 2x, DeMorgan'sche Regel 2x, Doppelte Verneinung
            \item K\"urzungsregel
            \item DeMorgan'sche Regel
            \item Idempotenz
            \item Distributiv Gesetz
            \item DeMorgan'sche Regel
            \item DeMorgan'sche Regel, Implikationsregel
        \end{enumerate}
    \end{enumerate}
\end{document}
